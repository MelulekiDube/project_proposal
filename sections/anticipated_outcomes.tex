\section{Anticipated outcome}
\subsection{System}
For the network managers and general users to see past and near-realtime performance of the network, the platform will employ a rich web based dashboard that will display performance measurement graphs for different servers on the network.
\paragraph{}
The interface will make use of different graphs to represent relevant patterns in the data. For frequency distributions, the tool will make use of histograms. Line graphs will also be used to show how the network has been performing for a certain amount of time. The visualiser will use blue and white for its user interface colours as these are simple and common colours.
\paragraph{}
The visualizer will be interactive such that users will be able to customise certain parameters like time so as to view the specific data they are interested in. Administrators will also be able to generate visual reports of the network.
\paragraph{}
The dashboard will also include an option for users to see a visualisation showing their past network usage which might include type of content being browsed, amount of data used and how often they use a certain service e.g Facebook.

\subsection{Expected Project Impact}
Large networks carry a lot of packets that may not be related and this makes it difficult for network managers to spot anomalous behaviours in the network\cite{Ruan2018}.
\paragraph{}
Our project will include a visualiser that will clearly display data collected from the network traffic. The visualiser will employ graphs that will clearly show the administrator unusual behaviours that may be due to a potential cyber attack or a failure in the network.

\paragraph{}
For end users, the visualiser will make them aware of their present and past activity in the network in the form of graphs.
\subsection{Key Success Factors}
Success factors for the data visualizer:
\begin{itemize}
	\item end-users are able to easily and effectively see their data usage in the network.
	\item Network operators/managers can easily visualize activity in the networks. For example number of users in the network or any point of failure. 
\end{itemize}
Success factors for data collection and storage
\begin{itemize}
	\item Accurately collect data from end user's smartphones based on different criteria such as latency, throughput and signal strength.
	\item Data collected is properly stored in a database and can be easily queried for display.
\end{itemize}
Success factors for orchestration tool
\begin{itemize}
	\item Ability to schedule network measurements on demand or at a user defined time.
	\item Allow selection of different network metrics to run measurements on. 
\end{itemize} 


