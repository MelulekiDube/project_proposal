\section{Related Works}
Since we are creating a Quality of Services monitoring system, one aspect that is needed is the measurements and collection of data from the mobile phones. Therefore we looked at works that has been done in this regard. We looked into systems that collected data from phones and those that collected data from more complex nodes \cite{7523537, Shepard:2011:LMW:1925019.1925023}. 
\paragraph{}
One of the important works that was looked at summarized the challenges involved when collecting measurements from mobiles phones. These challenges included legal issues and power consumed by these types measurement systems \cite{Shepard:2011:LMW:1925019.1925023}. The Live-Lab systems \cite{Shepard:2011:LMW:1925019.1925023} looked into eliminating these challenges by providing a systems which was capable of running internet measurements on different iPhone 3 mobile devices. The implementation details of this system were provided and thus for our monitoring systems we can adapt more from the system.
\paragraph{}
From the work by \cite{7523537,8255998} Cassandra database was the database that was used for storing the collected for its fast loading and querying perks. We will therefore be looking into also using a Cassandra like database, that is a column store type of database.

\paragraph{}
Other works, included MONROE which is a platform for running large-scale measurements and experiments in operational Mobile Brodband networks\cite{7523537}.One of MONROE functionalities' is its ability to run real-time user defined measurements together with data analysis on a few selected nodes available across europe\cite{7523537}. This measurements were initiated through a user-friendly web graphical interface that remotely controlled hundreds of nodes scattered over four European countries, with the purpose of assessing network performance\cite{7523537}. In the implementation of the orchestration functionality we intend to replicate aspects of their design to enable remote triggering of nodes in the community network to collect user-defined measurements at scheduled time periods. 

\paragraph{}
In terms of anomaly detection, classification based systems are found to be the most accurate in the detecting intrusions in the network according to \cite{AHMED201619}\cite{6524462}. This is the case, due to the presence of several datasets available for evaluating network anomaly detection methods and systems for example NSL-KDD dataset \cite{6524462}. They are found to be accurate on common attacks such as Denial
of Service (DoS), User to Root (U2R), Remote to Local (R2L),
and Probe or Scan\cite{7307098}. In the case of new network attacks classification systems tend to be very ineffective as they can not accurately detect attacks they have not dealt with before\cite{AHMED201619}. Unsupervised methods on the other hand tend to perform well as they group data based on how similar they are and since normal network traffic tends to occur more than anomalous data, it can be detected relatively accurately\cite{8357700}.  
 