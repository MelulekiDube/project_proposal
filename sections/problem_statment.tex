\section{Problem Statement}
\subsection{Aims and Research Question}
Community networks are decentralized communication networks which are large scale and self organized, built by the people for the people\cite{Braem:2013:CRC:2500098.2500108}. These networks provide a sustainable solution to address the connectivity gaps that exist in underserved urban, remote, and rural areas around the world\cite{Braem:2015:AEQ:2830629.2830639}. Prominent Examples include Zenzeleni, Rhizomatica and Inethi which is our case study for the project.  

Inethi which is a community network seeks to work with communities to co-design a content sharing and services platform for community wireless networks\cite{inethi}. Its goal is to build more resilient communities by using information technologies to help its users tap into local creativity, innovation, and other resources, with an eye towards improvement of socio-economic status\cite{inethi}.

The aim of this project is to provide researchers and operators of the network with a tool they can use to perform measurements on the community network, also with the ability to schedule collection of network data at specified times. The project also aims to provide end users with the ability to view their data usage accurately in a very user friendly manner. All the functionality discussed previously will be collectively packaged together with a easy to use visual interface. The ultimate goal of the application is to use the data collected enhance quality of service, prevent attack on the network as well as contribute further to the research in community networks.

The main question that we will be seeking to answer with the project are as follows:
\begin{itemize}
	\item How do different measurement collection techniques compare to each other in terms of power efficiency and network efficiency. Where network efficiency focuses on the effects of collecting measurements on the network. Power efficiency looks into the effects on power utilization on of the measurement collection technique used.
	\item How effective are different measurement scheduling mechanisms on data collection 
	\item How efficient different network probing mechanisms are in network analysis 
    \item investigate the effect of co-design process for network based web application interface through usability evaluation
\end{itemize}

\subsection{Requirements}
Our research takes on the form of a software engineering project with members of the Inethi community network as the primary user of the application. The application needs to meet their requirements for it to be of use.
Requirements of the application are as follows 
\begin{itemize}
	\item Provide functionality to perform measurements on the network based on different user-defined fields such latency, signal strength or throughput.
	
	\item Enable the scheduling of network measurements either on demand or at specified time.  
	
	\item Detection of anomalies in the network based on the analysis of the data collected in the network.
	
	\item Creation of user profiles which can be use to track their data usage in the network. 
	
	\item Provide data visualization to both end users in the viewing of their user profiles as well to network operators in the monitoring of the network.
\end{itemize}
Thorough testing and prototyping is paramount, to ensure we meet the requirements, and to provide Inethi with a tool that they are satisfied with.
