\section{Problem Statement}
\subsection{Aims and Research Question}
The aim of this project is to provide researchers and operators of the network with a tool they can use to perform measurements on the community network, also with the ability to schedule collection of network data at specified times. The project also aims to provide end users with the ability to view their data usage accurately in a very user friendly manner. All the functionality discussed previously will be collectively packaged together in a easy to use visualization tool. The ultimate goal of the application is to use the data collected enhance quality of service, prevent attack on the network as well as contribute further to the research in community networks.

The main question that we will be seeking to answer with the project are as follows:
\begin{itemize}
	\item It is a long established fact that a reader will be distracted by the readable content of a page when looking at its layout. 
	\item We want to look into the efficiency of different mechanisms for collections of measurement data. We will thus look into the different systems and compare these to determine the most efficient system if any. We will look into two main techniques for network measurement collection and these are, push-pull, scheduling and polling. 
	
	Push pull is a technique for measuring data where we have two entities in the system say the server and a mobile phone that we are collecting data from. Pulling then in this case means that the server requests for data from the mobile phone while pushing refers to the mobile phone then sending the data to the server when certain conditions that they are given are met \cite{article}. Polling in this case means that the we have devices which are collecting data also known as data producers always listening at fixed time intervals for new changes in the networks to record measurements and send data to the data consumers. That it constantly and repeatedly checks to see if anything in the network has changed and if so, run collects extra data that it will then send to the server. Scheduling means that we will have scheduled times for collecting  data and therefore the data producers will only collect measurements in the network on the scheduled times. 
	
	For each of the mentioned measurements collection technique we will measure their efficiency in terms of the following:
	\begin{itemize}
		\item How the network is affected by the measurements data collection. Taking network measurements will mean that we are utilizing the network links for collecting data.This can tend to have negative impacts on the network as there will be much more traffic on the network that is caused by these measurements. Therefore, the most efficient system will be one that causes minimum traffic on the network.
		\item Efficiency will also include a look into how the collection of network measurements will affect the user's mobile phones in terms of power consumption. Shepard, et al. suggested that interrupt driven logging was more efficient that polling systems \cite{Shepard:2011:LMW:1925019.1925023}. We will thus in this research look into how inefficient the polling system and compare it to the results that they found for interrupt driven logging.
	\end{itemize}
\end{itemize}

\subsection{Requirements}
Our research takes on the form of a software engineering project with members of the inethi community network as the primary user of the application. The application needs to meet their requirements for it to be of use.
\
Requirements of the application is as follows 
\begin{itemize}
	\item Provide functionality to perform measurements on the network based on different user-defined fields such latency, signal strength or throughput.
	
	\item Enable the scheduling of network measurements either on demand or at specified time.  
	
	\item Detection of anomalies in the network based on the analysis of the data collected in the network.
	
	\item Creation of user profiles which can be use to track their data usage in the network. 
	
	\item Provide data visualization to both end users in the viewing of their user profiles as well to network operators in the monitoring of the network.
\end{itemize}
Thorough testing and prototyping is paramount, to ensure we meet the requirements, and to provide Inethi with a tool that they are satisfied with.
