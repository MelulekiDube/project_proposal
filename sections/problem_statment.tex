\section{Problem Statement}
\subsection{Aims of the Project}
Community networks are decentralized communication networks which are large scale and self organized, built by the people for the people\cite{Braem:2013:CRC:2500098.2500108}. These networks provide a sustainable solution to address the connectivity gaps that exist in underserved urban, remote, and rural areas around the world\cite{Braem:2015:AEQ:2830629.2830639}. Prominent Examples include Zenzeleni, Rhizomatica and Inethi which is our case study for the project.  

Inethi which is a community network seeks to work with communities to co-design a content sharing and services platform for community wireless networks\cite{inethi}. Its goal is to build more resilient communities by using information technologies to help its users tap into local creativity, innovation, and other resources, with an eye towards improvement of socio-economic status\cite{inethi}.

The aim of this project is to provide researchers and operators of the network with a tool they can use to perform measurements on the community network, also with the ability to schedule collection of network data at specified times. The project also aims to provide end users with the ability to view their data usage accurately in a very user friendly manner. All the functionality discussed previously will be collectively packaged together with a easy to use visual interface. The ultimate goal of the application is to use the data collected enhance quality of service, prevent attack on the network as well as contribute further to the research in community networks.
\subsection{Requirements}
Our research takes on the form of a software engineering project with members of the Inethi community network as the primary user of the application. The application needs to meet their requirements for it to be of use.
Functional requirements of the application are as follows 
\begin{itemize} %to be presented as from the users perspective.
	\item Allow researchers perform measurements on the network based on different user-defined fields such latency, signal strength or throughput. %Allow researcher to perfrom measurements. Make it come from the users point of view
	\item Scheduling network measurements either on demand or at specified time. %change to schedule network measuremetns.
	\item Detect anomalies in the network due to network misconfiguration, network attacks, network failure. 
	%Detection of network anomalies due to any of the following network misconfiguration or network attacks, network failure.
	\item Allow users to maintain user profiles and track their network usage
	%Maintain user profiles/ 
	\item Provide data visualization to both end users in the viewing of their user profiles as well to network operators in the monitoring of the network.
	%Tp end user network managers and network operatos.
\end{itemize}
Non-functional requirements for the application include:
\begin{itemize}
	\item Should be secure in terms of the user data recorded.
	\item Should be reliable.
	\item Should be available for users all the time.
	\item Should  give correct results and graphs.
	\item Should be easily usable for all different users.
\end{itemize}
%non funtion 
Thorough testing and prototyping is paramount, to ensure we meet the requirements, and to provide Inethi with a tool that they are satisfied with.
%include non-functional requrements in this space.
\subsection{Research Questions}
In providing the above requirements, we will be answering the following questions:
\begin{itemize} %add quesiton on what the question looks at.
	%put the qurstion to be in italics.
	%question should end in questions marks
	\item \textit{How do different measurement collection techniques compare to each other in terms of power efficiency and network efficiency?} Where network efficiency focuses on the effects of collecting measurements on the network. Power efficiency looks into the effects on mobile phone power utilization of the measurement collection technique used.
	\item \textit{To what extent does the number of nodes probed as well as the frequency at which they are probed affect network efficiency}? This focuses mainly on the amount of overhead introduced in the network when collecting measurements from the network nodes 
	\item  \textit{Investigate the effectiveness of co-design process for network based web application interface through usability evaluation}? %put question in itelics
	
\end{itemize}