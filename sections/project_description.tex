 \section{Project Description}
This project will build a distributed platform for collecting and analysing Internet quality of service (QoS) metrics within the context of a wireless community network. The platform should be able to help detect sub-optimal or anomalous traffic flows, such as due to misconfiguration or network attacks.

To achieve this, the project will build a set of distributed Internet measurement units to gather Internet performance datasets in community networks. In addition, the monitoring system will incorporate feature selection mechanisms for detecting and correlating factors impacting network QoS.

The final output will be an intelligent QoS system that incorporates network measurements infrastructure and data processing pipeline. As a case study, this project will carried out within the context of iNethi Community Network, a localized content sharing and services platform being developed in Ocean View, a township in Cape Town.


\subsection{Project Significance}
Network measurement tools are now important to all users from administrators to general users. Without measurement platforms that provide a clear visualisation of network activity, administrators must scroll through long unorganized tables of data collected from the network to make sense of the activity in the network. This is an inefficient and time consuming task which deters the detection anomalous behaviour.

With a large number of people getting connected, users often want to know their network usage and their footprint on the internet. For example a user maybe interested in knowing the data they consumed over a specific amount of time, the services they use the most or how often they use the internet. Users also want this data delivered in a user friendly way hence the need for an elegant tool to visualise the data.

This project therefore aims at building a distributed platform for collecting and analysing internet quality of service(QoS) metrics within the context of a wireless community network. The platform should be able to help detect sub-optimal or anomalous traffic flows, such as due to misconfiguration or network attacks.

To achieve this, the project will build a set of distributed internet measurement units to collect internet performance data sets in community networks. The platform will also incorporate feature selection mechanisms for detecting and correlating factors impacting QoS.

\subsection{Issues and Difficulties}
There has been industrial and academic endeavours to characterise the internet through network measurements. However none of the prominent tools have been used in a significant way in developing regions especially in wireless community networks.

One of the challenges is that community networks are vastly different from known large common internet networks[1]. They are implemented using cheap hardware and networking methods. Most of them also have very few physical connections within the network. This gives them a unique architecture which poses a big challenge when it comes to implementing a traditional network measurement tool.

Community networks are also plagued with cyber security risks as user privacy may not be guaranteed. Network measurement tools have been suspected to lead to cyber security back doors in networks. 


