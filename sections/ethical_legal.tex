\section{Ethical And Legal Issues}
For this project, We have identified that the legal and ethical issues to consider are associated with data collection and data storage. Most specifically, the issues arising with collecting data from user phones as well as handling the data collected from the users.

\subsection*{Data Collection}
With Data collection, we have to worry of a number about things and these are:
\begin{itemize}
	\item What type of data needs to be collected. This may include going to specific users and getting to what data they consider private and would not want to leave their phones. Sensitive data that will need user's permissions will include their particular location that we will need for other measurements we take to make sense. 
	\item Since the system will need to keep track of the different applications that users will use. In attempts to measure which application used most of the data we will have to also consult first with the users to get their permission to do this. Furthermore, we will look into ways of anonymizing data collected. The aim is that when stored on the database, data will not be traceable back to the users. This solution will be adopted from the Live-Lab research project who accomplished anonymizing of data by hashing of user names of the users sending the data \cite{Shepard:2011:LMW:1925019.1925023}. We will adopt the same approach  and hash the messages that we are sending not the user names. The other data that users need to build their profile for network utilization statistics can be stored locally on the phone so that it is only accessible on their devices and no sensitive data leaves the users' phones.
\end{itemize}
\subsection*{Data Storage}
 We need to store data in a way that will make it almost impossible for anyone viewing the it to be able to track it to a particular user. That is to say, for each data item, no one can be able to detect the identity of a user by looking at the data. At the same time we need to know the different devices we are getting the data from so we can be able to keep track of the number of people using the system. This can be achieved by hashing the user name of the different users just before the data is saved on the database. Hashing will ensure that we have unique keys for the data we will be sending online and at the same time we can keep a count of the different keys to know how many people are using the application.

\paragraph{}
We will ensure that the user's sensitive data will not get to leave the users phone if necessary. 
