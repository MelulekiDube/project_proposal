\section{Ethical And Legal Issues}
For this project, We have identified that the legal and ethical issues to consider are associated with data collection and data storage. Most specifically the issues arising with collecting data from user phones as well as handling the data collected from the users.

\subsection*{Data Collection}
With Data collection we have to be worry of a number of things and these are:
\begin{itemize}
	\item What type of data needs to be collected. This may include going to specific users and getting to what data they consider private and would not want to leave their phones. Sensitive data that will need user's permissions will include their their particular location that we will need for other measurements we take to make sense. 
	\item Since the system will require to save keep track of the different application that the users use in attempts to measure what application used a lot of the data we will have to also consult first with the users to first get their permission on doing this. Also we want to also look into a way of anonymizing the data collected so that when stored on the database it will not be able to track the data back to particular users. This solution will be adopted from the Live-Lab research project who accomplished anonymizing of data by hashing of user names of the users sending the data \cite{Shepard:2011:LMW:1925019.1925023}. We feel we can adopt that and hash the message that we are sending on not the user names. The other data that users need to build their profile on their network utilization statistics can be stored locally on the phone so that it is only accessible on their phones and no data that is sensitive that leaves the users phones.
\end{itemize}
\subsection*{Data Storage}
The storage of data has already been hinted above. We need to store the data on the database in a way that will make it almost impossible for who ever is viewing the data to be able to track it to a particular user. That is for each data item no one can be able to detect the identity of a user by looking at the data. At the same time we need to know the different devices we are getting the data from so as we can be able to keep track of the number of people using the system. We have already stated that this can be achieved by hashing the user name of the different users just before the data is sent to the database. Hashing ensures that we will have uniques keys for the data that we will be sending online and at the same time we can keep a count of the different keys to know how many people are using the application.

\paragraph{}
We want to ensure that the user's data sensitive data will not get to leave the users phone if necessary. 
